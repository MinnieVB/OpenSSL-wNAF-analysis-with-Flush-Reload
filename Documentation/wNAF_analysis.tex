\documentclass[twocolumn]{article}
\usepackage{color}
\usepackage{soul}
\usepackage{cite}
\usepackage{array}
\usepackage{listings}
\usepackage{graphicx}
\usepackage{amsfonts}
\usepackage{amsmath, amsfonts, url, amssymb, graphics, algorithm2e,url} 
\usepackage{xspace}
%\usepackage{psnfss2e}
\usepackage{mathptmx}

\usepackage{url}
\newcommand{\starpar}[1]{\par{\footnotesize $\star$ \hl{#1}\par}}
\newcommand{\F}{{\mathbb F}}
\newcommand{\Z}{{\mathbb Z}}

\hyphenation{ec-dsa ec-dlp cu-rve}

\newcommand{\fl}{\textsc{Flu\-sh+\allowbreak Re\-load}\xspace}
\newcommand{\prpr}{\textsc{Prime+\allowbreak Probe}\xspace}

\newcommand{\myupcase}[1]{\uppercase{#1}}
%\newcommand{\myupcase}[1]{\textsc{#1}}

\lstset{basicstyle=\scriptsize,breaklines=true,breakindent=60pt,xleftmargin=20pt,numberstyle=\tiny,numbersep=5pt,language=C}

\begin{document}

\title{Working title: Extending the capabilities of the \fl Cache Side-channel Attack on OpenSSL}
%\author{Yuval Yarom \and Naomi Benger \and Nigel Smart}
%\institute{Yuval Yarom \and Naomi Benger \at The University of Adelaide, \email{yval@cs.adelaide.edu.au, naomi.benger@adelaide.edu.au}}

\maketitle

\begin{abstract}

\end{abstract}

\section{Introduction}
In~\cite{yarom_benger_2013} some of the issues related to using the attack when implementations used the sliding window for scalar multiplications instead of the Montgomery ladder... in this work we bypass these issues by using the LLL based techniques~\cite{nguyen03insecurity} to recover the nonce.

\subsection{Related Work}\label{sub:related}
This paper is not only a fantastic read but the authors are just tops too! \cite{yarom_benger_2013}

\section{Background}\label{sec:background}
In this section we present the relevant general background information about the attack and also the specific information required to understand the context of the example attack. 

\subsection{ECDSA}\label{sub:ecdsa}

The ElGamal Signature Scheme \cite{Elgamal85} is the basis of the US 1994 NIST standard, Digital Signature Algorithm (\myupcase{dsa}). The \myupcase{ecdsa} is the adaptation of one step of the algorithm from the multiplicative group of a finite field to the group of points on an elliptic curve. The main benefit of using this group as opposed to the multiplicative group of a finite field is that smaller parameters can be used to achieve the same security level \cite{koblitz87elliptic,miller85use} due to the fact that the current best known algorithms to solve the discrete logarithm problem in the finite field are sub-exponential and those used to solve the \myupcase{ecdlp} are exponential.
See Balasubramanian and Kob\-litz \cite{balasubramanian-koblitz}, Adleman and Demarrais~\cite{adelman-demarrais} and developments thereof for more details. 


{\bf{Parameters:}} An elliptic curve $E$ defined over a finite field $\F_{q}$; a point $G\in E$ of a large prime order $n$ (generator of the group of points of order $n$). Parameters chosen as such are generally believed to offer a security level of $\sqrt{n}$ given current knowledge and technologies. Parameters are recommended to be generated following the Digital Signature Standard~\cite{fips}. The field size $q$ is usually taken to be a large odd prime or a power of $2$. The implementation of OpenSSL uses both prime fields and $q=2^m$; the results in this paper relate to the binary field case.

{\bf{Public-Private Key pairs:}} The private key is an integer $d$, $1<d<n-1$ and the public key is the point $Q=dG$.
 Calculating the private key from the public key requires solving the \myupcase{ecdlp}, which is known to be hard in practice for the correctly chosen parameters.
 The most efficient currently known algorithms for solving the \myupcase{ecdlp} have a square root run time in the size of the 
group \cite{WienerZ98,GallantLV00}, hence the aforementioned security level.
\vspace{0.5cm}

Suppose Bob, with private-public key pair $\{d_B,Q_B\}$, wi\-shes to send a signed message $m$ to Alice, he follows the following steps:
\begin{enumerate}
\item Using an approved hash algorithm, compute $e=Hash(m),$ take $\bar{e}$ to be the leftmost $\ell$ bits of $e$ (where $\ell=\min(\log_2(q),$ bitlength of the hash$)$). 
\item\label{rand_element} Randomly select $k\leftarrow_R\Z_n$.
\item\label{scalar_mult} Compute the point $(x,y)=kG\in E$. 
\item Take $r=x\bmod n$; if $r=0$ then return to step \ref{rand_element}.
\item Compute $s=k^{-1}(\bar{e}+rd_B)\bmod n$; if $s=0$ then return to step \ref{rand_element}.
\item Bob sends $(m,r,s)$ to Alice.
\end{enumerate}
The message $m$ is not necessarily encrypted, the contents may not be secret, but a valid signature gives Alice strong evidence that the message was indeed sent by Bob. She verifies that the message came from Bob by 

\begin{enumerate}
\item checking that all received parameters are correct, that $r,s\in\Z_n$ and that Bob's public key is valid, that is $Q_B\neq \mathcal{O}$ and $Q_B\in E$ is of order $n$.
\item Using the same hash function and method as above, compute $\bar{e}$.
\item Compute $\bar{s}=s^{-1}\bmod n$.
\item Find the point $(x,y)=\bar{e}\bar{s}G+r\bar{s}Q_B$.
\item Verify that $r=x\bmod n$ otherwise reject the signature.
\end{enumerate}

Step \ref{rand_element} of the signing algorithm is of vital importance, inappropriate reuse of the random integer led to the highly publicised breaking of Sony PS3 implementation of \myupcase{ecdsa}. 
Knowledge of the random value $k$, a.k.a.\ the \textit{ephemeral key} or the \textit{nonce}, leads to knowledge of the secret key.
All values $(m,r,s)$ can be observed by an eavesdropper, $\bar{e}$ can be found from $m$, $r^{-1}\bmod n$ can be easily computed from $n$ and $r$, and if $k$ is discovered then an adversary can find Bob's secret key through the simple calculation $$d_B=(sk-\bar{e})r^{-1}.$$

Our attack targets Step~\ref{scalar_mult} of the OpenSSL implementation of \myupcase{ecdsa}.

\subsection{Performing scalar multiplication using wNAF}\label{sub:wnaf}
Scalar multiplication is a common operation in cryptography and in a number of incidences (such as step~\ref{scalar_mult} of \myupcase{ecdsa}) the scalar is intended to remain secret. This scalar multiplication is most efficiently performed using a double-and-add method (or the related right-to-left method) as outlined in Algorithm~\ref{d_and_a}.\\

\begin{algorithm}[htb]\label{d_and_a}
\SetAlgoLined
{\bf Input:} Point $P$, scalar $n$, $k$ bits\\
{\bf Output:} Point $nP$\\
$Q\gets \mathcal{O}$\\
 \For{$i$ from $k$ to $0$}{
  $Q\gets 2Q$\\
  \If{$n_i$ = 0}{
   $Q\gets Q+P$
   }
 }
 \caption{Double-and-add point scalar multiplication}
\end{algorithm}\vspace{-0.5cm}


%Double-and-add methods, though efficient, are vulnerable to side-channel attacks. The addition law for points on commonly used elliptic curves is not complete.
That is, the computation of $P+Q$ differs between the cases $P=Q$ and $P\neq Q.$ Consequently, it is possible to distinguish when the \texttt{if} in the  loop is executed and hence when a bit of~$n_i$ is~0.

In the context step~\ref{scalar_mult} of \myupcase{ecdsa} the scalar multiplication is performed on the same point every time the protocol is executed, only the random nonce changes. The wNAF method seeks to imporve efficiency by both exploiting this fact, and also by introducing more zeros into the representation of the scalar, so that additions are calculated more rarely. 

\subsubsection{NAF}

\subsubsection{Sliding Window}





***description of nWAF***


Instead of distinguishing additions from doublings, our attack identifies which arm of the \texttt{if} statement is taken.
The technique we use is described in the next section.


%\newpage

\subsection{The \fl attack}
\fl is a recently developed cache side-channel attack~\cite{yarom13flush}.
The attack exploits a weakness in the IA-32 and X86-64 processor architectures, which allows
processes to manipulate the cache of other processes.

Using the attack, a spy program can trace or monitor memory read and execute access of a victim program to shared memory pages.
The spy program only requires read access to the shared memory pages, hence pages containing binary code in executable files and
in shared libraries are susceptible to the attack.
By monitoring the victim access to specific locations in these pages, the spy program learns when the victim
executes the code in the monitored memory locations.
From this information the spy program can infer information on the data processed by the victim.

The spy program described in Yarom and Falkner~\cite{yarom13flush} uses the \fl attack to retrieve
the secret key from the GnuPG RSA decryption.
The spy program monitors the phases of the square-and-multiply exponentiation~\cite{gordon98survey} used by GnuPG.  
As these phases depend on the values of the bits of the exponent, monitoring them
allows the spy program to recover the secret exponent.

The attack operates by dividing time into slots.  
At the beginning of a time slot, the spy program flushes the monitored memory line from the cache of the processor.
At the end of the slot, the spy program loads data from the memory line.
Loading data from cached memory lines is significantly faster than loading them from memory.
Hence, by measuring the time it takes to load the data, the spy program can know whether the line is cached or not.
As the line is flushed at the beginning of the slot, having it cached at the end indicates that the processor accessed
the line during the time slot.

When the victim memory access overlaps the spy measurement, the spy will miss the access~\cite{yarom13flush}.
Consequently, increasing the time slot length reduces the portion of time the spy spends in
measurement and with it the probability of missing access.
On the other hand, the spy is unable to distinguish between multiple accesses to the same memory line in 
a single time slot.
It also cannot determine the order of memory accesses to different memory lines occurring in the same time slot.
Consequently, increasing the time slot reduces the attack's resolution.
Hence choosing the length of the time slot presents a tradeoff between the attack resolution
and the probability of missing a memory access.


\section{Attacking OpenSSL ECDSA}\label{sec:attack}
OpenSSL is one of the most commonly used  open-source cryptographic libraries.
It provides a set of cryptographic services, including both public key and symmetric encryption 
algorithms, and public key signature algorithms.

OpenSSL's implementation of \myupcase{ecdsa} uses one of two methods for performing the scalar multiplication; either the Montgomery ladder of sliding window. The implementation of the Montgomery ladder has been fully analysed in~\cite{yarom_benger_2013}, in the work presented below we demonstrate that implementations using the sliding window method are equally vulnerable to the \fl attack.


Listing~\ref{lst:openssl} shows the relevant section of the implementation of the Montgomery ladder in OpenSSL version 1.0.1e.
The bits of the multiplication scalar are stored in the word array \texttt{scalar->d},
where the word size is defined by the architecture, e.g.\ 32 bits for the IA-32 architecture and 64 bits for the X86-64 architecture.
The outer loop, at lines~268 to~286 traverses over the words representing the scalar.
The inner loop, at lines~271 to~284 traverses the bits in each word.
Line~273 tests the bit. 
For each bit the implementation executes a group add followed by a group double.
If the bit is set, the implementation uses lines~275 and~276.
For clear bits it uses lines~280 and~281.

\begin{lstlisting}[numbers=left,firstnumber=268,float=htb,caption=OpenSSL implementation of the wNAF sliding window,label=lst:openssl]
for (; i >= 0; i--)
{
    *** insert code
}
\end{lstlisting}

As the listing demonstrates, the implementation is regular: For each bit, the implementation executes exactly the same sequence of operations.
The only differences between set and clear bit are the lines that invoke these operations and the targets of these operations.
While this is a small difference, it is sufficient for mounting an attack that recovers 
the values of the bits.


Our spy program uses the \fl technique to monitor the execution of the \texttt{if}
statement in line~273.
We distinguish between executing the \texttt{then} and the \texttt{else} blocks of the \texttt{if}
statement.
This information reveals the value of the bit tested by the \texttt{if} statement.

\fl monitors execution by placing probes on shared memory lines.
For the attack to recover the bit values, it must distinguish between memory lines access sequences
resulting from set bits and those resulting from clear bits.
Achieving this depends on several factors: the mapping of source code to memory lines, 
the sequence of accesses to these memory lines when executing the code and 
\fl's ability to accurately capture the sequences.


The mapping of source lines to cache lines in our build of OpenSSL is depicted in Fig.~\ref{dgm:memory}.
The machine code created from source lines~273 to~282 covers the virtual memory address range 0x0812130C
to 0x081213e8.
This range spans four cache lines, marked $A$, $B$, $C$ and $D$.


%\begin{figure}[htb]
%\centering\includegraphics[width=\columnwidth]{images/memory}
%\caption{Mapping from source code to memory\label{dgm:memory}}
%\end{figure}


The minimum sequence of memory line accesses required for executing this code can now be constructed.
The \texttt{if} statement at line~273 is executed for each bit.  
The code of this statement is in memory line $A$, hence this memory line is accessed when processing of a bit starts.
For a set bit, the processing continues with source line~275, which maps to memory lines~$A$ and~$B$.
The actual call to the group add function occurs at address 0x08121347.
(See mark in Fig.~\ref{dgm:memory}.)
After a delay for computing the group add, execution continues in memory line~$B$ to process the return value and 
to invoke the group doubling function.
The group doubling function returns to memory line~$B$ and execution leaves the \texttt{if} body at memory line~$D$.

Hence, the sequence of memory line accesses required for a set bit is: $A$, $B$, \textit{add}, $B$, \textit{double}, $B$, $D$.
Similarly, for a clear bit, the sequence is: $A$, $C$, \textit{add}, $C$, $D$, \textit{double}, $D$.

Due to the limited temporal resolution of \fl, the attack can observe the order of memory accesses only
if they are sufficiently separated in time.
Hence, in the case of OpenSSL, the attack can only observe the order of memory accesses if they are separated by a call
to a group operation.
For example, when the bit is set, the attack cannot decide whether the access to memory line~$A$ precedes or follows the access
to memory line~$B$.
Similarly, when observed by \fl, memory accesses issued after the group double are merged with those 
issued at the start of processing the following bit.
Figure~\ref{dgm:temporal} shows the observable memory accesses when processing a set bit followed by
a clear bit.

%\begin{figure}[htb]
%\centering\includegraphics[width=\columnwidth]{images/temporal}
%\caption{Observable memory access over time (processing a set then  a clear bit)\label{dgm:temporal}}
%\end{figure}

The diagram also shows memory accesses issued by processor optimisations.
These optimisations pre-load memory lines into the cache to reduce the time the program waits for these lines.
For example, when the processor uses speculative execution~\cite{uht95disjoint}, it follows both arms of a conditional
branch before evaluating the condition.
When the condition is evaluated, the processor commits to the pre-processed computation of the correct arm,
disposing of the computation done for the other arm. 
In the case of OpenSSL this means that even before evaluating the bit, 
the processor may start processing both line~275 and line~280, triggering memory loads from memory lines~$A$, $B$ and $C$.

Another optimisation that can cause additional memory line access is spatial prefetching~\cite{intel12optimization}.
The processor pairs adjacent memory lines and tries to bring both memory lines into the cache
when there is a miss on one of the pair's lines.
For example, when there is a cache miss on memory line~$A$, the spatial prefetcher may attempt to prefetch memory line~$B$
and vice versa.

Consequently, as demonstrated in Fig.~\ref{dgm:temporal}, the memory lines accessed between computing
the group add and the group double can be used for recovering the value of the bit.
Probing any of lines~$A$ and~$B$ gives a positive indication of set bits.  
Probing either line~$C$ or~$D$ gives a positive indication of clear bits.
For our attack we probe memory lines~$B$ and~$D$.

Three limitations of the \fl attack affect its ability to capture the sequence of memory accesses.
The first is the attack temporal resolution which affects its ability to 
determine the order of accesses performed within a short time from each other.
The second limitation is the possibility of an overlap between the memory access and the probe which may result
in the attack missing the access.
The third limitation is the result of the interaction between the \fl attack and the processor
optimisation of cache use.  
In particular, the spatial prefetching optimisation implies that the attack cannot be used to probe two cache lines that form a pair,
because probing one of the lines in a pair triggers a prefetch of the other.



For OpenSSL, the attack resolution should be sufficiently high for the attack to be able to distinguish between 
memory accesses done before and after each bit and those done between the group add and group double operations of each bit.
This can be achieved by setting the time slot size to be less than the time it takes the victim to calculate the group double.
As group double calculations are faster than group add calculations, this ensures that the probed memory lines are flushed
when the victim computes the group add to be probed when the victim computes the group double.

The probability of an overlap, like the attack resolution, depends on the length of the time slot.
Longer time slots mean that the portion of time during which the spy probes is smaller and, therefore, 
the probability of an overlap is lower.

As predicted by Walter~\cite{walter04longer}, smaller keys are more resilient to the attack.
With smaller keys, group operations are shorter, forcing shorter time slots.
The shorter time slots lead to an increased probability of an overlap and with it of missing bits.

Missing memory accesses not only prevents the spy program from recovering the value of bits.
It may also result in the spy program losing the bit position in the scalar multiplication process.
To protect against this possibility, our attack also probes the first and last memory lines of the 
\texttt{gf2m\_Mdouble} function.
Probing these lines provides the spy program with additional information on the operation of the victim
and facilitates recovering the position of captured scalar bits.


The next section describes the details of our experimentation with the attack and its results.


\section{Experimental Setup and Results}\label{sec:results}

To test the \fl attack on OpenSSL we used an HP Elite 8300 running Fedora 18. As the OpenSSL package shipped with Fedora does not support \myupcase{ecc},
we used our own build of OpenSSL 1.0.1e. To facilitate the mapping from source lines to memory addresses we built OpenSSL with debugging symbols.
In real attack settings, the attacker will need to reverse engineer~\cite{cipsero10software} the OpenSSL library. For the experiment we used the OpenSSL \texttt{sect1571r1} curve. (NIST Binary-Curve B-571~\cite{fips}.)

With the selected curve, group add operations take 23,612 cycles on average. The first group double operation takes 6,552 cycles on average, whereas further group double operations take 11,962 cycles. Based on the discussion in Section~\ref{sec:attack}, we picked a slot length of 10,240 cycles.

\subsection{Full Recovery of the Nonce}\label{sub:full_nonce}

Given the higher proportion of missing bits, BSGS is not viable. 

With prob $p=2^{-n+1}$ we obtain the $n$ least significant bits. Then require $\frac{256}{n}$ signatures, thus by observing $\frac{256}{n2^{n-1}}$ signatures, we have enough to use the LLL based techniques~\cite{nguyen03insecurity}, ***implementation details, timings.***


\section{Discussion}\label{sec:discussion}


\subsubsection*{Implications}
Even if EC optimisations are present, the nature of the atttack would be able to distinguish bla bla... 


\subsubsection*{Mitigation}



\section{Conclusions and future work}\label{sec:conc}
This is the end of the world as we know it!

\section{acknowledgements}
The authors wish to thank Dr Katrina Falkner for the advice and support.

This research was performed under contract to the Defence Science and Technology Organisation (DSTO) Maritime Division, Australia.



\cite{nakamoto--bitcoin}

\bibliographystyle{plain}
\bibliography{wnaf_bib}

\end{document}

